\documentclass[a4paper]{article} 
\usepackage{polski}
\usepackage[utf8]{inputenc} 
\usepackage[OT4]{fontenc} 
\usepackage{graphicx,color}
\usepackage{url} 
\usepackage{hyperref}
\usepackage[a4paper,left=3cm,right=3cm,top=3cm,bottom=3cm,headsep=1.2cm]{geometry}
\usepackage{float} 

\title{\textbf{Sekwencjonowanie łańcuchów DNA}}
\date{}
\author{Paweł Lampe \and Jakub Szwachła}

\begin{document}

\maketitle

\section{Wstęp}

\section{Opis algorytmów}

\subsection{Problem 3: Sekwencjonowanie łańcuchów DNA z błędami negatywnymi i pozytywnymi}      %% fix a bit

Dla przypadku ogólnego przygotowano prosty algorytm zachłanny. Problem
zamodelowano w postaci grafu Łysowa, w którym każdy wierzchołek
odpowiada jednemu słowu ze spektrum. Wagi krawędzi odpowiadają
przesunięciu etykiet względem siebie. W każdym kroku podążamy krawędzią
o najmniejszej wadze.

Do algorytmu wprowadzono następnie ulepszenie. W drugiej wersji
algorytmu istnieje możliwość rozszerzania ścieżki z obu stron.

%% skomentować poprawę wyników

\subsection{Problem 1: Sekwencjonowanie łańcuchów DNA z błędami negatywnymi}
Przypadki z błędami wyłącznie negatywnymi można wewnętrznie podzielić na dwie podgrupy:
\begin{itemize}
\item braki w spektrum
\item błędy wynikające z powtórzeń
\end{itemize}
Z racji powyższego podziału, zaproponowana przez nas heurystyka wykonuje przed przystąpieniem do działania preprocesing.

Zlicza ona mianowicie ilość spójnych składowych w grafie. Jeśli jest dokładnie jedna to pewnym jest fakt, że w grafie
mamy do czynienia tylko i wyłącznie z błędami negatywnymi wynikającymi z powtórzeń. W przeciwnym razie, w grafie
mogą występować dowolne błędy z podgrup wymienionych powyżej.

Preprocesing wykonujemy więc celem zadecydowania jaką podjąć strategię.

Jeśli mamy do czynienia z jedną spójną składową w grafie, czyli na pewno z błędami wynikającymi z powtórzeń, sytuacja jest
stosunkowo prosta. Wystarczy zadbać aby zaszedł warunek konieczny istnienia ścieżki Eulera dla grafu skierowanego. 
Oznacza to, że trzeba znaleźć te wierzchołki których stopień wejściowy jest różny od stopnia wyjściowego, gdzie żadna
z tych liczb nie jest zerem. Gdy zostaną odnalezione, należy je odpowiednio ze sobą połączyć. Dokonać tego można poprzez
wybranie dowolnego z tych wierzchołków, oraz wyszukanie komplementarnego metodą BFS.

Alternatywną sytuacją jest przypadek, gdy w grafie jest więcej niż jedna spójna składowa. %% opisać IF!=1

Jeżeli w powyższych metodach wszystko przebiegnie bezproblemowo, wystarczy poszukać wierzchołka początkowego dla ścieżki
Eulera (stopień wejściowy = 0 oraz stopień wyjściowy = 1) oraz odszukać tą ścieżkę celem podania rozwiązania.

Algorytm ten jest jak widać zorientowany na dwa przypadki. %% napisać o FAILach


\end{document}
