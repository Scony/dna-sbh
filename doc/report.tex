\documentclass[a4paper]{article} 
\usepackage{polski}
\usepackage[utf8]{inputenc} 
\usepackage[OT4]{fontenc} 
\usepackage{graphicx,color}
\usepackage{url} 
\usepackage{hyperref}
\usepackage[a4paper,left=3cm,right=3cm,top=3cm,bottom=3cm,headsep=1.2cm]{geometry}
\usepackage{float} 

\title{Sekwencjonowanie łańcuchów DNA}
\date{}
\author{Paweł Lampe \and Jakub Szwachła}

\begin{document}

\maketitle

\section{Wstęp}

\section{Opis algorytmów}

\subsection{Problem 3: Sekwencjonowanie łańcuchów DNA z błędami
negatywnymii pozytywnymiymi}

Dla przypadku ogólnego przygotowano prosty algorytm zachłanny. Problem
zamodelowano w postaci grafu Łysowa, w którym każdy wierzchołek
odpowiada jednemu słowu ze spektrum. Wagi krawędzi odpowiadają
przesunięciu etykiet względem siebie. W każdym kroku podążamy krawędzią
o najmniejszej wadze.

Do algorytmu wprowadzono następnie ulepszenie. W drugiej wersji
algorytmu istnieje możliwość rozszerzania ścieżki z obu stron.

%% skomentować poprawę wyników

\subsection{Problem 1: Sekwencjonowanie łańcuchów DNA z błędami
negatywnymi}


\end{document}
